\section{Generalidades}
Los sensores urbixCell-WiFi, a partir de ahora llamados nodos,
se basan en la detección de paquetes WiFi.
Cuando un celular tiene su sistema WiFi activo
envía paquetes en forma periódica que pueden ser detectados 
identificando la fuente o celular, mac address,
identificador único de dada dispositivo que pueda ser conectado a una red.
La periodicidad con la cual estos celulares
enví­an estos paquetes depende de muchos factores como 
la marca, el modelo, el nivel de batería, 
la configuración del usuario, el uso de apps, y otros)
A partir de esta información, que por el momento estamos reduciendo a solamente el caso de probe requests (P-R)
cada nodo/sensor construye clusters con el objeto de
comprimir la información y comenzar a definir entidades que tengan un sentido relacionado al movimiento peatonal.
El análisis de la información adquirida por los nodos supone que
hay una fracción $f_c$ de las personas que se quiere estudiar que
tienen celulares con WiFi, una fracción $f_a$ de estos que cuentan con
el sistema WiFi activo y enviando P-R, y una fracción $f_d$ de estos últimos que el nodo fue capaz de detectar
dada la problemática.
De esta forma, la proporción de personas que se detectan es:
\[
f = f_c \, f_a \, f_d
\]
Se busca conocer cual es la proporción de celulares detectados $f$
y  comprender el impacto de los efectos abajo listados:
\begin{enumerate}
\item Problemática. La cantidad $f_d$ depende fuertemente del comportamiento peatonal.
Es de esperar que en un entorno donde los peatonas pasan rápidamente (como puede ser un pasillo)
$f_d$ será menor que en los casos donde la gente
se quedó un largo tiempo como puede ser el caso de un restaurante.

\item Layout de la locación. El efecto de las paredes y otros
elementos que puedan tener influencia sobre la intensidad de la señal detectada

\item Características del parque de celulares, $f_c$.

\item El tipo y proporción de celulares $f_c$, como así el tipo de uso que
se le da $f_a$ tendrá impacto en la proporción final de celulares detectados.

\item Estacionalidad. Es importante evaluar si $f$
presenta algún tipo de modulación estacional (horaria, diaria, semanal)
que influya en $f_a$ o $f_c$.

\item Variables demográficas $f_c$.
\item Existencia de redes WiFi $f_a$ en la locación de estudio,
ya que una vez que un celular está conectado a una red, este
deja de enviar p-r
\item Otros.
\end{enumerate}

En casos donde no se puede determinar se busca conocer las cotas y/u orden de magnitud de estas proporciones.

