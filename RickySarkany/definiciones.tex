\section{Definiciones}

Llamamos cluster a un conjunto de detecciones P-R de 
una misma mac que asumimos que corresponden a una misma persona en una misma situación. 
De esta forma se define como cluster una unidad compuesta por una 
secuencia de P-R (pulsos) de una misma mac address, con 
$t_1$, 
mí­nimo intervalo de tiempo entre clusters o 
máximo intervalo de tiempo entre pulsos, y 
$t_2$, 
máximo intervalo de tiempo entre pulsos o 
mí­nimo tiempo de separación entre clusters. 
Como consecuencia de esta definición, la longitud temporal de un cluster 
es $\ge t_1$ y los intervalos entre clusters es $\ge t_2$.

Definimos
$f$, la proporción de personas que se detectan;
$P$, la cantidad de personas en la zona de detección;
$\alpha$ el error de tipo I, o falso positivo, que es la probabilidad
de detectar un cluster que no corresponde a la variable en estudio;
$\beta$ el error de tipo II, o falso negativo, que es la probabilidad
no detectar un cluster que corresponde a la variable en estudio;
$M$, la cantidad de clusters detectados,
$M_c$, cantidad de macs diferentes que son consideradas para la determinación de los clusters de interés.
$M\big|^p_t$ es la cantidad de clusters filtrados por potencia $p$ y tiempo de detección $t$

\[
M = M\big|^{p>-\infty}_{t\ge 0}
\]

\[
M_c = M\big|^{p>\bar{p}}_{t\ge 0}
\]

\begin{equation}
\bar{f} =
\frac
{
M\big|
^{p \ge \bar{p}}
_{t \ge \bar{t}}
}
{C}
\end{equation}

es el mejor estimador de del factor $f$ de detección en el pasillo
usando el mínimo error relativo excluyendo el 5\% superior e inferior de los casos
Sea $f_i$ el factor calculado con el $i$-simo percentil de la muestra

\[
(1 - \alpha) P = (1 - \beta) f M
\]

